
\documentclass{article}
\usepackage{amsmath}
\usepackage{amssymb}

\begin{document}

\section*{Lecture Title}
\subsection*{Description of the lecture}

\begin{enumerate}
\item A set $A$ is a subset of $B$, $ACB$, if $a \in A \implies a \in B$
\item Two sets are equal, $A=B$, if $A \subseteq B$ and $B \subseteq A$
\item A is a proper subset of $B$, $ASB$, if $ACB$ and $A \neq ASB$
\item Set building notation $\{ x \mid P(x) \}$
\end{enumerate}

\begin{enumerate}
\item $N=\{1,2,3,4,\ldots\}$
\item $Z=\{0,1,-1,-2,\ldots\}$
\item $Q=\{m \div n \mid m,n \in Z, n \neq 0\}$
\item $R = \{\text{real numbers} \cup \{\text{irrational numbers}\} \}$ along with irrationals like $\pi, \sqrt{2}, \ldots$
\item $N=\{2m-1 \mid m \in N\}$
\item $C=\{1,3,5,\ldots\}$
\end{enumerate}

\begin{enumerate}
\item The union of $A,B$ is the set $A \cup B = \{x \mid x \in A \text{ or } x \in B\}$
\item The intersection of $A,B$ is the set $A \cap B = \{x \mid x \in A \text{ and } x \in B\}$
\item The set $P(x)$
\end{enumerate}

\end{document}
