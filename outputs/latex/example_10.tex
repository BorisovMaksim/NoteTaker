\documentclass{beamer}
\usepackage{fancyvrb}
\usepackage{amsmath}
\usepackage{amssymb}

\begin{document}

\begin{frame}{}
  \begin{block}{Title of the Lecture}
    Description of the lecture
  \end{block}
  \begin{columns}
    \column{0.5\textwidth}
      <1-> A set $A$ is a subset of $B$, $ACB$, if $a \in A \implies a \in B$
      <2-> Two sets are equal, $A=B$, if $A \subseteq B$ and $B \subseteq A$
      <3-> A is a proper subset of $B$, $ASB$, if $ACB$ and $A \neq ASB$
      <4-> Set building notation $\{ x \mid P(x) \}$

    \column{0.5\textwidth}
      <1-> 
        \begin{enumerate}
          \item $N=\{1,2,3,4,\ldots\}$
          \item $Z=\{0,1,-1,-2,\ldots\}$
          \item $Q=\{m \div n \mid m,n \in Z, n \neq 0\}$
          \item $R = \{\text{real numbers} \cup \{\text{irrational numbers}\} \}$ along with irrationals like $\pi, \sqrt{2}, \ldots$
          \item $N=\{2m-1 \mid m \in N\}$
          \item $C=\{1,3,5,\ldots\}$
        \end{enumerate}
      <2-> The union of $A,B$ is the set $A \cup B = \{x \mid x \in A \text{ or } x \in B\}$
      <3-> The intersection of $A,B$ is the set $A \cap B = \{x \mid x \in A \text{ and } x \in B\}$
      <4-> The set $P(x)$
  \end{columns}
\end{frame}

\end{document}